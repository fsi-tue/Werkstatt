% Preamble
% ---
\documentclass[a4paper]{article}

% Packages
% ---
\usepackage{amsmath} % Advanced math typesetting
\usepackage[utf8]{inputenc} % Unicode support (Umlauts etc.)
\usepackage[ngerman]{babel} % Change hyphenation rules

\usepackage{amssymb}
\usepackage{hyperref} % Add a link to your document
\usepackage{graphicx} % Add pictures to your document
\usepackage{tabularx}
\newcolumntype{C}[1]{>{\centering\arraybackslash}p{#1}}
\usepackage{listings} % Source code formatting and highlighting
\usepackage[inline]{enumitem}
\usepackage{fullpage} %weiniger abstand zu den seiten
%Code
\usepackage{color}
\usepackage{colortbl}
\usepackage{textcomp}
\definecolor{listinggray}{gray}{0.9}
\definecolor{lbcolor}{rgb}{0.9,0.9,0.9}
\lstset{
	backgroundcolor=\color{lbcolor},
	tabsize=4,
	rulecolor=,
	basicstyle=\scriptsize,
	upquote=true,
	aboveskip={1.5\baselineskip},
	columns=fixed,
	showstringspaces=false,
	extendedchars=true,
	breaklines=true,
	prebreak = \raisebox{0ex}[0ex][0ex]{\ensuremath{\hookleftarrow}},
	frame=single,
	showtabs=false,
	showspaces=false,
	showstringspaces=false,
	identifierstyle=\ttfamily,
	keywordstyle=\color[rgb]{0,0,1},
	commentstyle=\color[rgb]{0.133,0.545,0.133},
	stringstyle=\color[rgb]{0.627,0.126,0.941},
}

\begin{document}

\author{Anleitung und Sicherheitsinformationen} % The authors name
\title{GERÄTENAME}
\date{\today{}} % Sets date you can remove \today{} and type a date manually
\maketitle{} % Generates title
\section{Nutzungsberechtigung}
WIE DARF DAS GERÄT GENUTZT WERDEN? BSP.:
Dieses Gerät darf nur mit einer gültigen Einweisung und Sicherheitsbelehrung genutzt werden. Bei Fragen oder Problemen wendet euch an einen Betreuer.
\section{Gefahrenhinweise}
\begin{center}
	\begin{tabular}{C{5cm}C{5cm}C{5cm}}
		\includegraphics[width=2.5cm]{hot-surface.png} & \includegraphics[width=2.5cm]{hand-injury.png} & \includegraphics[width=2.5cm]{toxic.png}\\
		\textbf{Achtung heiße Oberfläche!} & \textbf{Achtung Quetschgefahr!} & \textbf{Achtung Giftig!}
	\end{tabular}
\end{center}
WOHER KOMMT DIE GEFAHR? WAS IST HEIß? WAS IST GIFITG? WAS IST ZU BEACHTEN? \\
WAS MACHT MAN WENN ETWAS SCHIEF GEHT?!

\section{Technische Daten}
 \begin{tabular}{|l|l|}
 	\hline
 	Hersteller & BCN3D\\
 	\hline
	Leistung & 240W \\
	SONSTIGE INFOS & HIER\\
	\hline
\end{tabular}
\newpage
\section{Bedienung}
\subsection{Allgemeines}
ALLGEMEINE INFORMATIONEN
WAS DARF OHNE BETREUER GEMACHT WERDEN? WAS NUR MIT
\subsection{*-VORBEREITUNG}
\begin{itemize}
	\item ALLE VORBEREITUNGSSCHRITTE
\end{itemize}
\subsection{WÄHREND DES *}
\begin{itemize}
	\item WAS IST WÄHRENDDESSEN ZU BEACHTEN
\end{itemize}
\subsection{NACH DEM *}
\begin{itemize}
	\item WIE RÄUMT MAN AUF?
\end{itemize}
\end{document}
